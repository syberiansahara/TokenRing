\documentclass{article}
\usepackage{listings}
\usepackage{amsmath}
\usepackage{amssymb}
\usepackage[usenames,dvipsnames]{color}
\usepackage{comment}
\usepackage[T2A]{fontenc}
\usepackage[utf8]{inputenc}
\usepackage[russian]{babel}

\definecolor{Orange}{cmyk}{0.0,0.56,0.58,0.11}
\definecolor{Gray}{gray}{0.5}
\lstset{
    language=Java,
    basicstyle=\ttfamily,
    keywordstyle=\color{Orange},
    commentstyle=\color{Gray},
    captionpos=b,
    breaklines=true,
    breakatwhitespace=false,
    showspaces=false,
    showtabs=false,
    numbers=left,
}


\title{Token Ring}
\author{Марина Белялова}
\date{26 ноября 2017}

\begin{document}

\begin{center}


\textsc{\Large Московский Физико-Технический Институт \\ (Государственный университет)}\\[1cm]
\textsc{\normalsize Кафедра банковских информационных технологий}\\[4cm]
{ \huge \bfseries Token Ring \\[1cm] }
\textnormal{\normalsize Выполнила cтудентка 285 гр. Марина Белялова}\\[10cm]
\textnormal{Москва, 2017}
\end{center}

%=============================================================

\newpage

\tableofcontents
%=============================================================

\newpage
\section{Описание задачи}
В данной работе описана реализация модели сетевого протокола Token Ring на языке Java. Целью данной работы являлось исследование зависимости характеристик latency и throughput от числа нод и загруженности нод, а так же поиск варианта оптимизации работы для недогруженного и перегруженного режимов передачи пакетов.

\begin{itemize}
\item Система состоит из N пронумерованных от 0 до N-1 нод. Ноды упорядочены по порядковому номеру. После ноды N-1 следует нода 0, т.е. ноды формируют кольцо. 
\item Соседние в кольце ноды могут обмениваться пакетами. Обмен возможен только по часовой стрелке. 
\item Каждая нода, получив пакет от предыдущего, отдает его следующему.
\item Пакеты не могут обгонять друг друга.
\end{itemize}

\section{Основные сущности и соответствующие классы}
Сообщения, которые передаются в системе, реализованы в классе Message. Сообщение имеет размер (здесь был выбран размер 3кБ) и при отправке фиксирует в себе время. Содержит в себе логическую величину
\lstinline|hasBeenDelivered.|

Класс Frame реализует сущность пакета, который может пребывать в двух состояниях: \lstinline|isToken() = true|, т.е. фрейм пустой и не содержит в себе сообщение, либо фрейм содержит сообщение, которое может быть доставлено и не доставлено. Когда фрейм представляет собой токен, любая нода, желающая отправить сообщение, может использовать этот фрейм.

Класс Node реализует узел сети. Каждая нода имеет очередь \lstinline|Queue<Message> pendingMessages = new ConcurrentLinkedQueue<>()| сообщений, ожидающих отправки, и очередь входящих фреймов \lstinline|Queue<Frame> enqueuedFrames = new ConcurrentLinkedQueue<>()|. Node содержит в себе логику обработки входящих фреймов:

\input Classes/Node

В методе \lstinline|void handleIncomingMessage(Frame currentFrame)| успешность доставки сообщения имеет распределение Бернулли с вероятностью успеха $p$. Конкретные значения в экспериментах указаны в соответствующих разделах.



\section{Основные метрики}
Для измерения времени использовалась системная функция \lstinline|System.nanoTime()|.


\section{Исследование зависимости latency и throughput от числа нод и загруженности нод}

Вероятность успеха доставки сообщения $p=0.85$.

\section{Исследование зависимости latency и throughput от token holding time при различных значениях загруженности ноды}

Вероятность успеха доставки сообщения $p=0.75$.

\newpage
% \input Classes/Message

% \input Classes/Frame

% \input Classes/Node

% \input Classes/MessageGenerator

% \input Classes/Launcher

\input Classes/LauncherExtract







\end{document}
